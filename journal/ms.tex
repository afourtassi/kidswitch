\documentclass[english,,man,floatsintext]{apa6}
\usepackage{lmodern}
\usepackage{amssymb,amsmath}
\usepackage{ifxetex,ifluatex}
\usepackage{fixltx2e} % provides \textsubscript
\ifnum 0\ifxetex 1\fi\ifluatex 1\fi=0 % if pdftex
  \usepackage[T1]{fontenc}
  \usepackage[utf8]{inputenc}
\else % if luatex or xelatex
  \ifxetex
    \usepackage{mathspec}
  \else
    \usepackage{fontspec}
  \fi
  \defaultfontfeatures{Ligatures=TeX,Scale=MatchLowercase}
\fi
% use upquote if available, for straight quotes in verbatim environments
\IfFileExists{upquote.sty}{\usepackage{upquote}}{}
% use microtype if available
\IfFileExists{microtype.sty}{%
\usepackage{microtype}
\UseMicrotypeSet[protrusion]{basicmath} % disable protrusion for tt fonts
}{}
\usepackage{hyperref}
\hypersetup{unicode=true,
            pdftitle={Continuous developmental change can explain discontinuities in word learning},
            pdfauthor={Abdellah Fourtassi, Sophie Regan, \& Michael C. Frank},
            pdfkeywords={word learning, cognitive development, computational modeling},
            pdfborder={0 0 0},
            breaklinks=true}
\urlstyle{same}  % don't use monospace font for urls
\ifnum 0\ifxetex 1\fi\ifluatex 1\fi=0 % if pdftex
  \usepackage[shorthands=off,main=english]{babel}
\else
  \usepackage{polyglossia}
  \setmainlanguage[]{english}
\fi
\usepackage{graphicx,grffile}
\makeatletter
\def\maxwidth{\ifdim\Gin@nat@width>\linewidth\linewidth\else\Gin@nat@width\fi}
\def\maxheight{\ifdim\Gin@nat@height>\textheight\textheight\else\Gin@nat@height\fi}
\makeatother
% Scale images if necessary, so that they will not overflow the page
% margins by default, and it is still possible to overwrite the defaults
% using explicit options in \includegraphics[width, height, ...]{}
\setkeys{Gin}{width=\maxwidth,height=\maxheight,keepaspectratio}
\IfFileExists{parskip.sty}{%
\usepackage{parskip}
}{% else
\setlength{\parindent}{0pt}
\setlength{\parskip}{6pt plus 2pt minus 1pt}
}
\setlength{\emergencystretch}{3em}  % prevent overfull lines
\providecommand{\tightlist}{%
  \setlength{\itemsep}{0pt}\setlength{\parskip}{0pt}}
\setcounter{secnumdepth}{0}
% Redefines (sub)paragraphs to behave more like sections
\ifx\paragraph\undefined\else
\let\oldparagraph\paragraph
\renewcommand{\paragraph}[1]{\oldparagraph{#1}\mbox{}}
\fi
\ifx\subparagraph\undefined\else
\let\oldsubparagraph\subparagraph
\renewcommand{\subparagraph}[1]{\oldsubparagraph{#1}\mbox{}}
\fi

%%% Use protect on footnotes to avoid problems with footnotes in titles
\let\rmarkdownfootnote\footnote%
\def\footnote{\protect\rmarkdownfootnote}


  \title{Continuous developmental change can explain discontinuities in word
learning}
    \author{Abdellah Fourtassi\textsuperscript{1}, Sophie Regan\textsuperscript{1},
\& Michael C. Frank\textsuperscript{1}}
    \date{}
  
\shorttitle{Continuous Development of Word Learning}
\affiliation{
\vspace{0.5cm}
\textsuperscript{1} Department of Psychology, Stanford University}
\keywords{word learning, cognitive development, computational modeling}
\usepackage{csquotes}
\usepackage{upgreek}
\captionsetup{font=singlespacing,justification=justified}

\usepackage{longtable}
\usepackage{lscape}
\usepackage{multirow}
\usepackage{tabularx}
\usepackage[flushleft]{threeparttable}
\usepackage{threeparttablex}

\newenvironment{lltable}{\begin{landscape}\begin{center}\begin{ThreePartTable}}{\end{ThreePartTable}\end{center}\end{landscape}}

\makeatletter
\newcommand\LastLTentrywidth{1em}
\newlength\longtablewidth
\setlength{\longtablewidth}{1in}
\newcommand{\getlongtablewidth}{\begingroup \ifcsname LT@\roman{LT@tables}\endcsname \global\longtablewidth=0pt \renewcommand{\LT@entry}[2]{\global\advance\longtablewidth by ##2\relax\gdef\LastLTentrywidth{##2}}\@nameuse{LT@\roman{LT@tables}} \fi \endgroup}


\usepackage{lineno}

\linenumbers
\usepackage{tipa}
\usepackage[sortcites=false,sorting=none]{biblatex}

\authornote{ Abdellah Fourtassi

Department of Psychology

Stanford University

50 Serra Mall

Jordan Hall, Building 420

Stanford, CA 94301

Correspondence concerning this article should be addressed to Abdellah
Fourtassi, Postal address. E-mail:
\href{mailto:afourtas@stanford.edu}{\nolinkurl{afourtas@stanford.edu}}}

\abstract{
``Cognitive development is often characterized in term of
discontinuities, but these discontinuities can sometimes be apparent
rather than actual and can arise from continuous developmental change.
To explore this idea, we use as a case study the finding by Stager and
Werker (1997) that children's early ability to distinguish similar
sounds does not automatically translate into word learning skills. Early
explanations proposed that children may not be able to encode subtle
phonetic contrasts when learning novel word meanings, thus suggesting a
discontinuous/stage-like pattern of development. However, later work has
revealed (e.g., through using simpler testing methods) that children do
encode such contrasts, thus favoring a continuous pattern of
development. Here we propose a probabilistic model describing how
development may proceed in a continuous fashion across the lifespan. The
model accounts for previously documented facts and provides new
predictions. We collected data from preschool children and adults, and
we showed that the model can explain various patterns of learning both
within the same age and across development. The findings suggest that
major aspects of cognitive development that are typically thought of as
discontinuities, may emerge from simpler, continuous mechanisms.''


}

\usepackage{amsthm}
\newtheorem{theorem}{Theorem}[section]
\newtheorem{lemma}{Lemma}[section]
\theoremstyle{definition}
\newtheorem{definition}{Definition}[section]
\newtheorem{corollary}{Corollary}[section]
\newtheorem{proposition}{Proposition}[section]
\theoremstyle{definition}
\newtheorem{example}{Example}[section]
\theoremstyle{definition}
\newtheorem{exercise}{Exercise}[section]
\theoremstyle{remark}
\newtheorem*{remark}{Remark}
\newtheorem*{solution}{Solution}
\begin{document}
\maketitle

\section{Introduction}\label{introduction}

Cognitive development is sometimes characterized in terms of a
succession of discontinuous stages (Piaget, 1954). Although intuitively
appealing, stage theories can be challenging to integrate with theories
of learning, which typically posit that knowledge and skills improve
incrementally with experience. Indeed, one of the central challenges of
cognitive development has been to explain transitions between stages
which appear to be qualitatively different (Carey, 2009).

Nevertheless, at least in some cases, development may only appear to be
stage-like. This appearance can be due, for example, to the use of a
cognitively-demanding task which may mask learning, or to the use of
statistical thresholding (in particular, p-value \textless{} 0.05) which
can create a spurious dichotomy between success and failure in observing
a given behavior. In such cases, positing discontinuous stages is
unnecessary. Instead, a continuous model---involving similar
representations across the lifespan---may provide a simpler and more
transparent account of development.

We use a case study from word learning literature. Stager and Werker
(1997) first showed that children's early ability to distinguish similar
sounds does not automatically translate into word learning skills.
Indeed, though infants around 14-month old can distinguish similar sound
pairs such as \enquote{dih} and \enquote{bih}, they appear to fail in
mapping this pair to two different objects.

By around 17 m.o, children succeed in the same task (Werker, Fennell,
Corcoran, \& Stager, 2002). How does development proceed? The answer to
this question depends on the nature of the chidlren's word encoding in
the task. Early accounts assumed that children encode words in a rather
binary way: they either fail or succeed in attending to the relevant
phonetic details (simultaneously with the meanings). In the orginal
paper, Stager and Werker (1997) noted that \enquote{Infants of 14 months
fail to detect the same phonetic detail that they can easily detect in a
simple syllable discrimination task}. They argued that the high cogntive
demands of the task make children experience a limitation in the amount
of phonetic details they can access. In a similar vein, Pater et al.
(2004) argued that \enquote{when a contrast is first acquired, it is not
stable and can be lost under processing demands}. This binary account
suggested a discontinuous/stage-like pattern of development whereby
younger children fail to encode the contrastive phonetic detail, whereas
older children succeed.

Subsequent findings have investigated the implications of this
hypothesis in both infants and adults. First, if the demands of the task
totally empede the encoding of subtle phonetic contrasts, then infants
should fail in learning the mappings between the sounds and the objects,
regardless of how this learning is probed. However, 14-month-olds
succeed in this task under the same \emph{learning} conditions as in
Stager and Werker (1997), but when an easier \emph{testing} method is
used (Yoshida, Fennell, Swingley, \& Werker, 2009). Second, if the
minsmatch between sound distrimination and word learning is only related
to limited cognitive resources in infancy, then adults --- who have
greater attentianal and working memory capacity --- should always
succeed in mapping pairs of sounds they can discriminate to different
objects. Nonetheless, even adults show patterns learning that mirror
those shown by 14-month-olds (Pajak, Creel, \& Levy, 2016; White, Yee,
Blumstein, \& Morgan, 2013).

This new set of evidence points towards another scenario, where the
representations are encoded in a probabilistic (rather than binary) way,
and where development is continuous, rather than stage-like (see also
Swingley, 2007). On this account, correct representations are learned
early in development, but these representations are encoded with higher
uncertainty in younger children, leading to \emph{apparent} failure in
relatively demanding tasks. Development is a continuous process whereby
the initial noisy representations become more precise. Crucially, in a
probabilsitic account, more precise representation do not mean they
perfect, thus accounting for the fact that even adults show low accuracy
learning when the sounds are subtle, e.g., non-native (Pajak et al.,
2016).

We provide an intuitive illustration of how such an account explains
patterns of learning and development in Figure \ref{fig:illus}. We
observe low accuracy in word learning when the perceptual distance
between the labels is small relative to the uncertainty with which these
labels are encoded. For example, in Stager and Werker's original
experiment, children are supposed to associate label 1 (\enquote{bih})
and label 2 (\enquote{dih}) with object 1 and object 2, respectively.
Though infants could learn that the label \enquote{bih} is a better
match to object 1 than \enquote{dih}, they could still judge the sound
\enquote{dih} as a plausible instance of the label \enquote{bih}, thanks
to the relatively large uncertainty of the encoding, and this confusion
leads to \enquote{failure} in the recognition task. According to this
account, accuracy in word learning improves if we increase either the
perceptual distinctiveness of the stimuli (e.g., through using
different-sounding labels), or the precision of the encoding itself
(e.g., across development).

Building on this intuition, the current work proposes a probabilistic
model, which we use to both account for previous experimental findings,
and to make new predictions that have not been tested before. Using new
data collected from both preschool children and adults, we show that the
model can explain various patterns of learning both within the same age
and across development.

\begin{figure}

{\centering \includegraphics[width=300px]{figs/illustration} 

}

\caption{An illustration of the probabilistic/continuous account using simulated data. A word is represented with a distribution over the perceptual space (indicated in red or blue). When the uncertainty of the representation is large relative to the distance between the stimuli (top panel), an instance of the red category (indicated with a star) could also be a plausible instance of the green category, hence the low recognition accuracy score. The accuracy increases when the stimuli are less similar (left panel), or when the representation are more precise (right panel).}\label{fig:illus}
\end{figure}

\section{Model}\label{model}

\subsection{Probabilistic structure}\label{probabilistic-structure}

Our model consists of a set of variables describing the general process
of spoken word recognition in a referential situation. These variables
are related in a way that reflects the simple generative scenario
represented graphically in Figure \ref{fig:model}. When a speaker utters
a sound in the presence of an object, the observer assumes that the
object \(o\) activated the concept \(C\) in the speaker's mind. The
concept prompted the corresponding label \(L\). Finally, the label was
physically instantiated by the sound \(s\).

\begin{figure}

{\centering \includegraphics[width=300px]{figs/model} 

}

\caption{Graphical representation of our model. Circles indicate random variables (shading indicates observed variables). The squares indicate fixed model parameters.}\label{fig:model}
\end{figure}

A similar probabilistic structure was used by Lewis and Frank (2013) to
model concept learning, and by Hofer and Levy (2017) to model spoken
word learning. However, the first study assumed that the sounds are
heard unambiguously, and the second assumed the concepts are observed
unambiguously. In our model, we assume that both labels and concepts are
observed with a certain amount of perceptual noise, which we assume, for
simplicity, is captured by a normal distribution:

\[ p(o | C) \sim  \mathcal{N}(\mu_C, \sigma^2_C) \]

\[ p(s| L) \sim  \mathcal{N}(\mu_L, \sigma^2_L) \]

Finally, we assume there to be one-to-one mappings between concepts and
labels and that observers have successfully learned these mappings
during the exposure phase: \[
P(L_i|C_j) = 
\begin{cases}
  1 & \text{if  }  i=j \\  
  0  & \text{otherwise  }
\end{cases}
\]

\subsection{Inference}\label{inference}

The learner hears a sound \(s\) and has to decide which object \(o\)
provides an optimal match to this sound. To this end, they must compute
the probability \(P(o|s)\) for all possible objects. This probability
can be computed by summing over all possible concepts and labels:
\[P(o|s)=\sum_{C,L} P(o, C, L| s) \propto \sum_{C,L} P(o, C, L, s) \]

The joint probability \(P(o, C, L, s)\) is obtained by factoring the
Bayesian network in Figure \ref{fig:model}:
\[P(o,C,L,s) = P(s|L)P(L|C)P(C|o)P(o) \]

which can be transformed using Bayes rule into:

\[P(o,C,L,s) = P(s|L)P(L|C)P(o|C)P(C) \]

Finally, assuming that the concepts' prior probability is uniformly
distributed\footnote{This is a reasonable assumption in our particular case given the similarity of the concepts used in each naming situation in our experiment.},
we obtain the following expression, where all conditional dependencies
are now well defined:

\begin{equation}
P(o|s) = \frac{\sum_{C,L} P(s|L)P(o|C)P(L|C)}{\sum_{o} \sum_{C,L} P(s|L)P(o|C)P(L|C)}
\end{equation}

\subsection{Task and model
predictions}\label{task-and-model-predictions}

\begin{figure}[t]

{\centering \includegraphics[width=3.75in]{figs/task} 

}

\caption{An overview of the task used in this study.}\label{fig:task}
\end{figure}

We use the model to predict performance in the word learning task
introduced by Stager and Werker (1997), with a two-alternative forced
choice as in Yoshida et al. (2009). In this task, participants are first
exposed to the association between pairs of nonsense words (e.g.,
\enquote{lif}/\enquote{neem}) and pairs of objects. The word-object
associations are introduced sequentially. After this exposure phase,
participants perform a series of test trials. In each of these trials,
one of the two sounds is uttered (e.g., \enquote{lif}) and participants
choose the corresponding object from the two alternatives. An overview
of the task is shown in Figure \ref{fig:task}.

We used Equation 1 and the probability distributions defined above to
obtain the exact analytical expression for the probability of accurate
responses \(p(o_T | s)\) (target object \(o_T\) given a sound \(s\)) in
the simple case of two-alternative forced choice in the testing phase of
our experimental task:

\begin{equation}
P(o_T|s)= \frac{1 + e^{-(\Delta s^2/2\sigma_L^2+ \Delta o^2/2\sigma_C^2)}}{1 + e^{-(\Delta s^2/2\sigma_L^2+ \Delta o^2/2\sigma_C^2)}+ e^{-\Delta s^2 /2\sigma_L^2} + e^{-\Delta o^2 /2\sigma_C^2 }}
\end{equation}

Figure \ref{fig:simulation} show simulations of the predicted accuracy
(Expression 2) as a function of the distinctiveness parameters
(\(\Delta s\) and \(\Delta o\)) and the precision parameters, i.e., the
variances of the distributions \(p(s| L)\) and \(p(o | C)\). To
understand the qualitative behavior of the model, we assumed for
simplicity that the precision parameter has similar values in both
distributions, i.e., \(\sigma =\sigma_C \approx \sigma_L\) (but we will
allow those parameters to vary independently in the rest of the paper).

\begin{figure}[htbp]
\centering
\includegraphics{ms_files/figure-latex/simulation-1.pdf}
\caption{\label{fig:simulation}The predicted probability of accurate
responses in the testing phase as a function of stimuli distinctiveness
\(\Delta s\) and \(\Delta o\) and representation precision \(\sigma\)
(For clarity, we assume here that \(\sigma\)=\(\sigma_C\)=\(\sigma_L\)).
Dashed line represents chance.}
\end{figure}

The simulations explain some previously documented facts, and make new
predictions:

\begin{enumerate}
\def\labelenumi{\arabic{enumi})}
\item
  For fixed values of \(\Delta o\) and \(\sigma\), the probability of
  accurate responses increases as a function of \(\Delta s\). This
  pattern accounts for the fact that similar sounds are generally more
  challenging to learn than different sounds for both children (Stager
  \& Werker, 1997) and adults (Pajak et al., 2016).
\item
  For fixed values of \(\Delta s\) and \(\Delta o\), accuracy increases
  when the representational uncertainty (characterized with \(\sigma\))
  decreases. This fact may explain development, i.e., younger children
  have noisier representations (see Swingley, 2007; Yoshida et al.,
  2009), which leads to lower word recognition accuracy, especially for
  similar-sounding words.
\item
  For fixed values of \(\Delta s\) and \(\sigma\), accuracy increases
  with the visual distance between the semantic referents \(\Delta o\).
  This is a new prediction that our model makes. Previous work studied
  the effect of several bottom-up and top-down properties in
  disambiguating similar sounding words (e.g., Fennell \& Waxman, 2010;
  Rost \& McMurray, 2009; Thiessen, 2007), but to our knowledge no
  previous study in the literature tested the effect of the visual
  distance between the semantic referents.
\end{enumerate}

\section{Experiment}\label{experiment}

In this experiment, we tested participants in the word learning task
introduced above (Figure \ref{fig:task}). More precisely, we explored
the predictions related to both distinctiveness and precision. Sound
similarity (\(\Delta s\)) and object similarity (\(\Delta o\)) were
varied simultaneously in a within-subject design. Two age groups
(preschool children and adults) were tested on the same task to explore
whether development can be characterized with the uncertainty
parameters, \(\sigma_C\) and \(\sigma_L\). The experiment, sample size,
exclusion criteria and the model's main predictions were pre-registered.

\subsection{Methods}\label{methods}

\subsubsection{Participants}\label{participants}

We planned to recruit a sample of N=60 children ages 4-5 years from the
Bing Nursery School on Stanford University's campus. Here we report data
from n=55 children. An additional n=35 children participated but were
removed from analyses because they were not above chance on the catch
trials due to the challenging nature of our procedure (see below). We
also planned to recruit a sample of N=60 adults on Amazon Mechanical
Turk. Data from n=26 participants were excluded due to low scores on the
catch trials (n=26 ) or because they were familiar with the non-English
sound stimuli we used in the adult experiment (n=0), yielding a final
sample of n=74.

\subsubsection{Stimuli and similarity
rating}\label{stimuli-and-similarity-rating}

The sound stimuli were generated using the MBROLA Speech Synthesizer
(Dutoit, Pagel, Pierret, Bataille, \& Van der Vrecken, 1996). We
generated three kinds of nonsense word pairs which varied in their
degree of similarity to English speakers: 1) \enquote{different}:
\enquote{lif}/\enquote{neem} and \enquote{zem}/\enquote{doof}, 2)
\enquote{intermediate}: \enquote{aka}/\enquote{ama} and
\enquote{ada}/\enquote{aba}, and 3) \enquote{similar} non-English
minimal pairs: \enquote{ada}/\enquote{a\textipa{d\super h}a} (in hindi)
and \enquote{a\textipa{Q}a}/\enquote{a\textipa{\textcrh}a} (in arabic).

As for the objects, we used the Dynamic Stimuli javascript
library\footnote{https://github.com/erindb/stimuli} which allowed us to
generate objects in four different categories: \enquote{tree},
\enquote{bird}, \enquote{bug}, and \enquote{fish}. These categories are
supposed to be naturally occurring kinds that might be seen on an alien
planet. In each category, we generated \enquote{different},
\enquote{intermediate} and \enquote{similar} pairs by manipulating a
continuous property controlling features of the category's shape (e.g,
body stretch or head fatness).

In a separate survey, \(N=20\) participants recruited on Amazon
Mechanical Turk evaluated the similarity of each sound and object pair
on a 7-point scale. We scaled responses within the range {[}0,1{]}. Data
are shown in Figure \ref{fig:stim}, for each stimulus group. These data
will be used in the models as the perceptual distance of sound pairs
(\(\Delta s\)) and object pairs (\(\Delta o\)).

\includegraphics{ms_files/figure-latex/unnamed-chunk-6-1.pdf}
\#\#\#Design

Each age group saw only two of the three levels of similarity described
in the previous sub-section: \enquote{different} vs.
\enquote{intermediate} for preschoolers, and \enquote{intermediate} vs.
\enquote{similar} for adults. We made this choice in light of pilot
studies showing that adults were at ceiling with \enquote{different}
sounds/objects, and children were at chance with the \enquote{similar}
sounds/objects. That said, this difference in the level of similarity is
accounted for in the model through using the appropriate perceptual
distance used in each age group (Figure \ref{fig:stim}).

\begin{figure}[h]

{\centering \includegraphics{ms_files/figure-latex/stim-1} 

}

\caption{Distances for both sound and object pairs from an adult norming study. Data represent Likert values normalized to [0,1] interval. Error bars represent 95\% confidence intervals.}\label{fig:stim}
\end{figure}

To maximize our ability to measure subtle stimulus effects, the
experiment was a 2x2 within-subjects factorial design with four
conditions: high/low sound similarity crossed with high/low visual
object similarity. Besides the 4 conditions, we also tested participants
on a fifth catch condition which was similar in its structure to the
other ones, but was used only to select participants who were able to
follow the instructions and show minimal learning.

\subsubsection{Procedure}\label{procedure}

Preschoolers were tested at the nursery school using a tablet, whereas
adults used their own computers to complete the same experiment online.
Participants were tested in a sequence of five conditions: the four
experimental conditions plus the catch condition. In each condition,
participants saw a first block of four exposure trials followed by four
testing trials, and a second block of two exposure trials (for memory
refreshment) followed by an additional four testing trials. The length
of this procedure was demanding, especially for children, but we adopted
a fully within-subjects design based on pilot testing that indicated
that precision of measurement was critical for testing our experimental
predictions.

In the exposure trials, participants saw two objects associated with
their corresponding sounds. We presented the first object on the left
side of the tablet's screen simultaneously with the corresponding sound.
The second sound-object association followed on the other side of the
screen after 500ms. For both objects, visual stimuli were present for
the duration of the sound clip (about 800ms). In the testing trials,
participants saw both objects simultaneously and heard only one sound.
They completed the trial by selecting which of the two objects
corresponded to the sound. The object-sound pairings were randomized
across participants, as was the order of the conditions (except for the
catch condition which was always placed in the middle of the testing
sequence). We also randomized the on-screen position (left vs.~right) of
the two pictures on each testing trial.

\subsection{Results}\label{results}

\begin{figure}[h]

{\centering \includegraphics{ms_files/figure-latex/allData-1} 

}

\caption{Accuracy of novel word recognition as as a function of the sound distance, the object distance, and the age group (preschool children vs. adults). We show both experimental results (solid lines) and model predictions (dashed lines). Error bars represent 95\% confidence intervals.}\label{fig:allData}
\end{figure}

Experimental results are shown in Figure \ref{fig:allData} (solid
lines). We first analyzed the results using a mixed-effects logistic
regression with sound distance, object distance and age group as fixed
effects, and with a maximal random effects structure (allowing us to
take into account the full nested structure of our data) (Barr, Levy,
Scheepers, \& Tily, 2013). We found main effects for all the fixed
effects in the regression. For the sound distance, we obtained
\(\beta =\) 0.52 (\(p\) \textless{} 0.001), replicating previous
findings. For object distance, we found \(\beta =\) 0.83 (\(p\)
\textless{} 0.001), and this finding confirms the new prediction of our
model. Finally, for the age group, we obtained \(\beta =\) 0.76 (\(p\)
\textless{} 0.001), showing that performance improves with age. In
addition, we found two-way interactions between sound distance and age
(\(\beta =\) 0.15, \(p\) = 0.17) and between object distance and age
(\(\beta =\) 0.45, \(p\) \textless{} 0.001).

We next fit our model (using Equation 2) to the participants' responses
in each age group. The values of \(\Delta s\) and \(\Delta o\) were set
based on data from the similarity judgment task (Figure \ref{fig:stim}).
Thus, the model has two degrees of freedom for each group, i.e.,
\(\sigma_C\) and \(\sigma_L\). Figure \ref{fig:allData} (dashed lines)
shows the predictions. The model captures the qualitative patterns in
both age groups: starting from a low accuracy recognition when both the
sound and object distances are small, the model correctly predicts an
increase in accuracy when either the sound distance or the object
distance increases. Further, accuracy is correctly predicted to be
maximal when both the sound and object distances are high.

The values of the parameters were as follows. Children had a
label-specific uncertainty of \(\sigma_S =\) 0.83 {[}0.64,
1.02{]}\footnote{All uncertainty intervals in this paper represent 95\% Confidence Intervals.},
and a concept-specific uncertainty of \(\sigma_C =\) 0.31 {[}0.11,
0.51{]}. Adults had a label-specific uncertainty of \(\sigma_S =\) 0.12
{[}0.12, 0.13{]}, and a concept-specific uncertainty of \(\sigma_C =\)
0.17 {[}0.16, 0.18{]}. As predicted, the uncertainty parameters were
larger for children than they were for adults, showing that the
probabilistic representations becomes more refined (that is, \(\sigma\)
becomes smaller) across development. The developmental effect was more
important for the label-specific uncertainty.

The models explained the majority of the variance in the participants'
mean responses (\(R^2=\) 1, for the combined adult and children's data).
To investigate whether the model's predictive power was due to
overfitting, we fit a simplified version with only one degree of freedom
(i.e., a single variance common to both sounds and objects). This
single-variance model explained as much variance in the mean responses
(\(R^2=\) 0.95). It also captured the main qualitative patterns (graph
not shown), suggesting that the explanatory power of the model is
largely due to its structure, rather than its degrees of freedom.

An unexpected outcome was that adult participants deviated slightly from
the model's numbers: While the model predicted accuracy to be more
sensitive to object distance when sound distance is higher (and
vice-versa), adult participants showed the opposite pattern. This
deviation is an interesting starting point for future work as it may
suggest that participants are more likely to pay attention to and
integrate additional sources of information when ambiguity is higher.

\section{General Discussion}\label{general-discussion}

This paper explored the idea that some seemingly stage-like patterns in
cognitive development can be characterized in a continuous fashion. We
used as a case study the seminal work of Stager and Werker (1997)
showing a discrepancy between children's speech perception abilities and
their word learning skills. While much of the previous investigation of
this finding has been interested in the source of this discrepancy, here
we have explored how it could arise from continuous developmental change
in perceptual uncertainty.

Building on some previous discussions (e.g., Swingley, 2007; Yoshida et
al., 2009), we proposed a model where perceptual stimuli are encoded
probabilistically. We tested the model's predictions against data
collected from preschool children and adults and we showed that
developmental changes in word-object mappings can indeed be
characterized as a continuous refinement (i.e., uncertainty reduction)
in qualitatively similar representations across the life span.

The model made a new prediction to which we tested experimentally:
Learning similar words is not only modulated by the similarity of their
phonological forms, but also by the visual similarity of their semantic
referents. More generally, since visual similarity is an early
organizing feature in the semantic domain (e.g., Wojcik \& Saffran,
2013), our finding suggests that children may prioritize the acquisition
of words that are quite distant in the semantic space. This suggestion
is supported by recent findings based on the investigation of early
vocabulary growth (Engelthaler \& Hills, 2017; Sizemore, Karuza, Giusti,
\& Bassett, 2018).

One limitation of this work is that the model was fit to data from
children at a relatively older age (4-5 years old) than what is
typically studied in the literature (14-18 month-old). We selected this
older age group to optimize the number and precision of the experimental
measures (both are crucial to model fitting). Data collection involved
presenting participants with several trials across four conditions in a
between-subject design. It would have been challenging to obtain such
measures with infants.

In sum, this paper proposes a model that accounts for the development of
an important aspect of word learning. Our account suggests that the
developmental data can be explained based on a continuous process
operating over similar representations across development, suggesting
developmental continuity. We used a case from word learning as an
example, but the same idea might apply to other aspects of cognitive
development that are typically thought of as stage-like (e.g.,
acquisition of a theory of mind). Computational models, such as the one
proposed here, can help us investigate the extent to which such
discontinuities emerge due to genuine qualitative changes and the extent
to which they reflect the granularity of the researchers' own
measurement tools.

\vspace{1em}

\fbox{\parbox[b][][c]{14cm}{\centering All data and code for these analyses are available at\ \url{https://github.com/afourtassi/networks}}}
\vspace{1em}

\section{Acknowledgements}\label{acknowledgements}

This work was supported by a post-doctoral grant from the Fyssen
Foundation, NSF \#1528526, and NSF \#1659585.

\section{Disclosure statement}\label{disclosure-statement}

None of the authors have any financial interest or a conflict of
interest regarding this work and this submission.

\section{References}\label{references}

\setlength{\parindent}{-0.5in} \setlength{\leftskip}{0.5in}

\hypertarget{refs}{}
\hypertarget{ref-barr2013}{}
Barr, D., Levy, R., Scheepers, C., \& Tily, H. (2013). Random effects
structure for confirmatory hypothesis testing: Keep it maximal.
\emph{Journal of Memory and Language}, \emph{68}(3).

\hypertarget{ref-carey2009}{}
Carey, S. (2009). \emph{The origin of concepts}. Oxford University
Press.

\hypertarget{ref-dutoit1996}{}
Dutoit, T., Pagel, V., Pierret, N., Bataille, F., \& Van der Vrecken, O.
(1996). The mbrola project: Towards a set of high quality speech
synthesizers free of use for non commercial purposes. In
\emph{Proceedings of ICSLP} (Vol. 3). IEEE.

\hypertarget{ref-engelthaler2017}{}
Engelthaler, T., \& Hills, T. T. (2017). Feature biases in early word
learning: Network distinctiveness predicts age of acquisition.
\emph{Cognitive Science}, \emph{41}.

\hypertarget{ref-fennell2010}{}
Fennell, C., \& Waxman, S. (2010). What paradox? Referential cues allow
for infant use of phonetic detail in word learning. \emph{Child
Development}, \emph{81}.

\hypertarget{ref-hofer2017}{}
Hofer, M., \& Levy, R. (2017). Modeling Sources of Uncertainty in Spoken
Word Learning. In \emph{Proceedings of the 39th Annual Meeting of the
Cognitive Science Society}.

\hypertarget{ref-lewis2013}{}
Lewis, M., \& Frank, M. (2013). An integrated model of concept learning
and word-concept mapping. In \emph{Proceedings of the annual meeting of
the cognitive science society} (Vol. 35).

\hypertarget{ref-pajak2016}{}
Pajak, B., Creel, S., \& Levy, R. (2016). Difficulty in learning
similar-sounding words: A developmental stage or a general property of
learning? \emph{Journal of Experimental Psychology: Learning, Memory,
and Cognition}, \emph{42}(9).

\hypertarget{ref-piaget1954}{}
Piaget, J. (1954). \emph{The construction of reality in the child}. New
York, NY, US: Basic Books.

\hypertarget{ref-rost2009}{}
Rost, G., \& McMurray, B. (2009). Speaker variability augments
phonological processing in early word learning. \emph{Developmental
Science}, \emph{12}.

\hypertarget{ref-sizemore2018}{}
Sizemore, A. E., Karuza, E. A., Giusti, C., \& Bassett, D. S. (2018).
Knowledge gaps in the early growth of semantic feature networks.
\emph{Nature Human Behaviour}, \emph{2}(9).

\hypertarget{ref-stager1997}{}
Stager, C., \& Werker, J. (1997). Infants listen for more phonetic
detail in speech perception than in word-learning tasks. \emph{Nature},
\emph{388}(6640).

\hypertarget{ref-swingley2007}{}
Swingley, D. (2007). Lexical exposure and word-form encoding in
1.5-year-olds. \emph{Developmental Psychology}, \emph{43}(2).

\hypertarget{ref-thiessen2007}{}
Thiessen, E. (2007). The effect of distributional information on
children's use of phonemic contrasts. \emph{Journal of Memory and
Language}, \emph{56}.

\hypertarget{ref-werker2002}{}
Werker, J., Fennell, C., Corcoran, K., \& Stager, C. (2002). Infants'
ability to learn phonetically similar words: Effects of age and
vocabulary size. \emph{Infancy}, \emph{3}.

\hypertarget{ref-white2013}{}
White, K., Yee, E., Blumstein, S., \& Morgan, J. (2013). Adults show
less sensitivity to phonetic detail in unfamiliar words, too.
\emph{Journal of Memory and Language}, \emph{68}(4).

\hypertarget{ref-wojcik2013}{}
Wojcik, E., \& Saffran, J. (2013). The ontogeny of lexical networks:
Toddlers encode the relationships among referents when learning novel
words. \emph{Psychological Science}, \emph{24}(10).

\hypertarget{ref-yoshida2009}{}
Yoshida, K., Fennell, C., Swingley, D., \& Werker, J. (2009).
14-month-olds learn similar-sounding words. \emph{Developmental
Science}, \emph{12}.


\end{document}
